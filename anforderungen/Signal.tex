\documentclass[11pt]{article}

\usepackage[latin1]{inputenc}
\usepackage{ngerman}

\title{Datenkontainer-Bibliothek}
\author{Dr.-Ing. Philipp Guttenberg}

\begin{document}
\maketitle

\section{Funktionale Anforderungen}

\begin{enumerate}
\item Rechnen mit Skalaren, ein- und mehrdimensionalen Datenelementen muss bis mindestens drei Dimensionen m\"oglich sein.
\item Ein Check auf konsistente Laenge aller Unterelementevektoren einer Dimension eines Mehrdimensionalen Kontainers w\"are erw\"unscht. 
\item Das Verwenden der Elemente in Rechnungen auf der Komando-Zeile sollte dem Rechnen mit Matrizen in Matlab nahe kommen.
\item Die Rechnung muss mittels Typsystem oder Laufzeitabfragen abgesichert werden in Bezug auf:
  \begin {enumerate}
  \item die korrekte Verwendung und Kombination von Kontainertypen (Signal, Distribution, ...) 
  \item die physikalische Verwendung und Kombination von pkysikalischen Energiefluss-Gr\"ossen.
  \item der korrekte Verwendung und Kombination von Dimensionen und Vektorl\"angen.
  \item die Absicherung sollte nach M\"oglichkeit auch in gemappten Funktionen erhalten bleiben.   
  \end{enumerate} 
\item Werte in Graphiken, Diagrammen und Reports m\"ussen mittels Energiefluss-Typ und Einheit dargestellt werden.
\item Das Anzeigeformat (printf) muss f\"ur jede Kombination von Energiefluss-Typ und Anzeige-Einheit eingestellt werden k\"onnen.
\item Die Wahl der Anzeigeformate muss leicht \"anderbar sein, d.h. an einer zentralen Stelle erfolgen.
\item Der Standard-Datentyp (momentan double) f\"ur Gleitkommazahlen soll w\"ahlbar sein, z.B. single oder ratio.
\item Der Datenkontainer soll in jeder Instanz u\"ber Plot- und Report-Funktionen verf\"ugen.
\item Der Datenkontainer sollte auch benutzer-definierte Datentypen Unterst\"utzen (die eventuell nicht unboxed sein k\"onnen).
\item Die Rechnung mit dem Datenkontainer sind idealerweise effizient, lassen sich gut parallelisieren und ermoeglichen auch die Rechnung auf GPU's. 

\end{enumerate}


\section{Anforderungen zur Implementierung}

\begin{enumerate}
\item Der Datenkontainer sollte verschiedene Datenstrukturen verwenden k\"onnen, 
\begin{enumerate}
\item um lazy Evaluation zu erm\"oglichen (Liste)
\item effiziente Vektoren zu unterstuetzen (Unboxed Vector)
\item benutzerdefinierte Daten verwenden zu k\"onnen (Boxed Vector)
\item die Verwendung des Datentyps Ratio zu unterst\"utzen
\item Insbesondere ist eine zentrale Umschaltung der Rechnung auf Ratio vorzusehen.
\item Die Umschaltung zwischen Boxed und unboxed Vektor kann nach Bedarf erfolgen.
\end{enumerate}
\end{enumerate}

\section{Anforderungen zu physikalischen Typen}

Zu Unterscheiden sind:

\begin{enumerate}
\item verschiedene Energieflussgr\"o\"ssen
\item Absolut und Delta-Werte bis mindestens dritter Potenz
\item Totale und partielle Gr\"ossen
\end{enumerate}

\subsection{Energieflussgr\"o\"ssen}
Die Unterscheidung der physikalischen Gr\"o\"ssen ist nicht genug, 
da Teilungsfaktoren und Wirkungsgrade dimensionslos sind. Zudem ist 
Energiefluss mit Mix und Teilungsfaktoren und gespeicherte Energie mit 
Mix und Teilungsfaktoren zu unterscheiden. 

TODO -- hier wird noch am Konzept gedacht  

\section{Anforderungen zur Rechnung}

Die \"ublichen Rechenoperationen sollten unterst\"utzt sein.
\begin{itemize}
\item Summe
\item Produkt
\item verschiedene Trigonometrische Fuktionen
\item Absolut und Signum, ..
\item Logische Operationen -- gr\"osser, kleiner, ..
\item Integral und Ableitung
\end{itemize}

\section{Anforderungen zu Vektor-Operationen}
Die \"ublichen Listen und Vektor-Operationen sollten unterst\"utzt sein.

\begin{itemize}
\item folds and zips
\item head and tail
\item indizierung einzelner Werte oder von subvectoren
\end{itemize}

\section{Anforderungen zu Kontainer-Typen}

Folgende Kontainertypen sollen unterschieden werden k\"onnen:

\begin{itemize}
\item Skalare -- einzelner unabh\"angiger Wert
\item Signale und deren Zeitsamples
\item Distributionen und deren Samples (eventuell auch Histogramme und Kurven)  
\item Testreihen und deren Samples
\item Eine Unterscheidung der jeweiligen verk\"urzten Delta-Vektoren, falls L\"angenpr\"ufung nicht m\"oglich sein sollte.
\end{itemize}

Folgende Rechnungen auf Kontainern soll m\"oglich sein:
\begin{itemize}
\item Multiplikation und Addition mit Skalaren (elementweise) 
\item elementweise Operationen auf von Kontainer-Paaren mit gleicher Gr\"osse (Dimensionen und L\"angen)
\item wiederholt elementweise Operationen zwischen einem n-dimensionalen Kontainer mit einem n+1 dimensionalen 
Kontainer mit ansonsten gleichen Dimensionen und L\"angen
\item Aufblasen eines n-dimensionalen Kontainers auf n+1 Dimensionen 
\end{itemize} 
... TODO 

\section{Anforderungen zur Ausgabe}
\begin{itemize}
\item Plots und Reports einzelner Kontainer sollten leicht Kombinierbar sein.
\end{itemize} 

\section{Anforderungen in Bezug auf Wiederverwendbarkeit}
\begin{itemize}
\item die Containerbibliothek sollte eine eigenst\"andige Bibliothek sein, interne Funktionen sollten weitgehend versteckt werden
\item Abh\"angigkeiten auf andere Bibliotheken sollten minimiert werden
\item Eine Kombinierbarkeit bzw. Erweiterbarkeit auf ein physikalisches Typisierung-Modul w\"aere sehr w\"unschenswert.
\item Die Abh\"angigkeit von einzelnen konkreten Datenstrukturbibliotheken (Vector) wird indealerweise minimiert.
\end{itemize} 

\section{Anforderungen in Bezug auf Lizenzen}
\begin{itemize}
\item idealerweise nur Bibliotheken mit BSD3 Lizenz verwenden
\end{itemize}

\end{document}
